\section{Analiza sistema}

Student pravi profil gde popunjava osnovne podatke o sebi, bira oblasti interesovanja i postavlja dodatne filtere. Na osnovu izabranih oblasti prikazuju se objave na početnoj strani. Takođe, ponuđeni
su različiti tipovi objava (pitanje, zanimljivost, video-tutorijal, ponuda i slično).
Ako korisnik to želi, na svim objavama može da komentariše.
Za svaku svoju objavu ili komentar, korisnici dobijaju (pozitivne ili negativne) poene koje im daju drugi korisnici mreže. Korisnik može
učitati sve svoje objave i komentare, koji su izlistani u odnosu na poene koje
su dobili. Kada korisnik pravi novu objavu ima za cilj da to uradi tako da
zainteresuje druge korisnike da procitaju ili pregledaju tu objavu i da mu
daju poene. Što vise objava i poena ima, to korisnik brže stiče poverenje
ostalih.
Ukoliko korisnika zanimaju dešavanja vezana za neku oblast, može lako da dođe do informacija i da se poveže sa drugim korisnicima. Na taj način stiču nove sagovornike sa kojima mogu da dele iskustva i informacije.

\subsection{Akteri}
\begin{itemize}
    \item Korisnici - u korisnike spadaju pretežno studenti, ali i svi zainteresovani za informisanje, razmenu znanja i komunikaciju sa ostalim korisnicima socijalne mreže. Oni mogu uređivati svoje profile, postavljati i komentarisati objave, odazivati se na oglase i učlanjivati u grupe.
    \item Administratori - zaduženi su za održavanje sistema i kontrolisanje sadržaja objava na mreži. Takođe imaju pravo da uklone neprimerene objave i deaktiviraju profile korisnicima koji zloupotrebljavaju usluge mreže.
    \item Organizacije - kompanije i obrazovne institucije koriste usluge mreže zarad promovisanja, kao i  pronalaženja zaposlenih ili stipendista.
\end{itemize}

\subsection{Dijagram konteksta i DTP dijagram}
\begin{figure}[h!]
    \centerline{\includegraphics[scale=0.8]{slike/dijagram_konteksta.png}}
    \caption{Dijagram konteksta}
    \label{fig:my_label}
\end{figure}

\clearpage

\begin{figure}[h!]
		\centerline{\includegraphics[scale=1.8]{slike/dtp.png}}
		\captionof{figure}{DTP dijagram}
\end{figure}

\clearpage